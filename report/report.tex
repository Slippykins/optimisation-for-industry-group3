\documentclass[a4paper,12pt]{article}
\usepackage{graphicx}
\usepackage{amsmath, amssymb}
\usepackage{hyperref}
\usepackage{geometry}
\usepackage{listings}
\usepackage{multirow}
\usepackage{multicol}
\usepackage{booktabs}
\usepackage{caption}  
\usepackage{tabularx}
\usepackage{amsfonts}

\geometry{margin=1in}

\title{MAST 90014 - Optimisation for Industry \\ Group Project 2025}
\author{{Dean Soste}, {Yitian Wang}, {Tony Yue}, \\
{Zhouyi Cheng}, {Abdullah Alsuwat}, {Sahar Bahalgardi}
}
\date{\today}

\begin{document}

\maketitle




\section{Introduction}\label{sec:introduction}

Distribution planning during promotional periods represents one of the most challenging problems in retail supply chain management.
When demand can surge to 10 to 100 times overnight, traditional inventory and transportation strategies often fail, leading to either costly stock shortages or excessive inventory holding costs.
This project examines how mixed-integer programming can optimise distribution decisions for electronics retailers during Black Friday, one of the year's most intense promotional period.

Black Friday has evolved from a single-day sales event into a week-long phenomenon that tests the limits of retail supply chains.
For electronics retailers like JB Hi-Fi and Harvey Norman, success during this period can determine annual profitability.
The challenge extends beyond simply having enough inventory - retailers must position the right products at the right locations while managing constrained transportation capacity and escalating logistics costs.

We model a small-scale distribution network comprising 5 warehouses serving 5 retail stores over a 7-day Black Friday period.
The network must distribute three representative products: high-value smartphones experiencing 100-fold day-on-day demand increases, headphones with concentrated Black Friday purchasing, and earphones facing obsolescence risks and heavy discounting.
These products capture the diversity of challenges retailers face, from managing extreme demand spikes to balancing inventory risks for promotional items.

The optimisation problem addresses two fundamental decisions: how to manage truck fleet capacity (comparing flexible daily rentals against cheaper weekly contracts) and how to transport products throughout the week to meet current and future demands.
With truck availability limited to 10 vehicles per day (due to driver availability and to restrict the problem space) and each truck having a fixed capacity, the model must balance transportation against inventory costs while ensuring product availability during Black Friday demand surges.



\section{Problem Definition}\label{sec:problem-definition}

We address the distribution planning problem faced by typical electronics retailers during Black Friday promotional periods.
A retailer operating multiple warehouses must efficiently distribute products to retail stores over a one-week period encompassing Black Friday, when demand can spike 10x just between two subsequent days.

\subsection{Problem Context}\label{subsec:problem-context}

The company manages a network of 5 warehouses serving 5 retail stores, distributing three representative categories of consumer electronics: high-value smartphones, mid-range headphones, and heavily discounted earphones.
We chose to limit the scope of our case study to three representative products as we wanted to hand-craft data for a more realistic, tailored problem.
Each product category exhibits distinct demand patterns during Black Friday week, with:

\begin{itemize}
    \item Smartphones experiencing almost no demand prior to Black Friday as consumers await heavy discounts followed by a sustained 10x demand over the weekend,
    \item Headphones showing a moderate increase in demand for Black Friday (4x), and
    \item Earphones showing a 100x in daily demand between Thursday and Black Friday.
\end{itemize}

The full demand profiles for these products can be seen in Table~\ref{tab:table2}.

\subsection{Key Decisions}\label{subsec:key-decisions}

The retailer must make two decisions:
\begin{itemize}
    \item \textbf{Total Trucks to Rent}: How many trucks to rent for the period, capped at a maximum of 10 per day.
Depending on the problem variant (see Problem Variants~\ref{subsec:problem-variants}), daily rentals may be fixed for the week or may be allowed to vary via daily rental at a premium.
    \item \textbf{Distribution Planning}: How many units of each product to ship from each warehouse to each retailer in each time period.
These decisions must account for truck capacity constraints, varying transportation costs between locations, and the trade-off between holding inventory and risking stock shortages (missed sales opportunities).
\end{itemize}

\subsection{Problem Characteristics}\label{subsec:problem-characteristics}

Several factors make this problem particularly challenging:

\begin{itemize}
    \item \textbf{Extreme Demand Volatility}: Demand for each good changes dramatically throughout the week, with massive spikes on Black Friday that tend to continue throughout the weekend.
    \item \textbf{Capacity Constraints}: The company faces a hard constraint of 10 trucks per day due to driver availability.
With each truck having a set capacity and products occupying different space, maximising the utilisation of each truck becomes a challenge.
    \item \textbf{Heterogeneity of Goods}: Retailers and transportation companies must deal with incredibly varied goods, with significantly different characteristics in terms of shape, size, fragility, value and obsolescence.
These characteristics heavily impact transportation considerations and the various costs faced by the retailer.
    \item \textbf{Cost Trade-offs}: The retailer must balance four competing cost components:
    \begin{itemize}
        \item Transportation costs varying by distance and by good, capturing the overall cost of transporting each good and their unique characteristics applicable to transportation, like fragility.
        \item Truck rental costs.
        \item Inventory holding costs, including obsolescence risk for discounted items, expressed as a percentage of the retail price of the good.
        \item Shortage costs, reflecting lost profit margins and calculated as a percentage of the retail price.
    \end{itemize}
\end{itemize}

\subsection{Simplifications and Assumptions}\label{subsec:simplifications-and-assumptions}

To make the problem tractable while maintaining practical relevance, we adopt several simplifications:

\begin{itemize}
    \item \textbf{Deterministic Demand}: This is an unreasonable assumption, but adding uncertainty demand make the problem hard to model and difficult to understand the results.

    \item \textbf{No Shipments Between Retailers}: While stores in the real world may ship goods between each other to smooth over shortages and held inventory, we remove this possibility to simplify the model.

    \item \textbf{Unlimited Warehouse Supply}: We assume that there is some other upstream logistics model that has optimised warehouse stock for the Black Friday week.

    \item \textbf{Single Transportation Mode}: It's reasonable to assume that a company may use trucks of different sizes, but we fix all our trucks to a single size for simplicity.

    \item \textbf{Representative products}: In order to keep things simple for this report, we only model three goods, and assume that these goods take up roughly 1\% of the total capacity of a 50m3 truck.
    We rationalise this by arguing the goods are representative of the full cargo of the truck, and hence our results generalise to the full capacity.
\end{itemize}


\subsection{Problem Variants}\label{subsec:problem-variants}

We analyse four instances, characterised by the rental period of the trucks and the prevailing daily rental costs.

\begin{itemize}
    \item \textbf{Truck Rental Strategy}:
    \begin{itemize}
        \item Weekly rental: Lower daily cost but requires committing to a fixed number of trucks for the full week.
        \item Daily rental: Higher cost but allows for the number of trucks to vary throughout the week, especially on Thursday/Friday to meet Black Friday demands.
    \end{itemize}

    \item \textbf{Truck Rental Costs}:
    \begin{itemize}
        \item High truck costs: Representing truck rental companies increasing prices due to high demand around the Black Friday period.
        \item Low truck costs: Representing regular pricing.
    \end{itemize}
\end{itemize}


These variants allow us examine if our optimal solution is sensitive to truck management and if so to quantify how much of an impact on bottom-lines this can have.

The data for these variants is available in Section~\ref{sec:data}, \textit{Data}.

\subsection{Relevance and Applications}\label{subsec:relevance-and-applications}

While framed around Black Friday, this problem represents a broader class of distribution challenges during promotional periods.
Similar patterns occur during:
\begin{itemize}
    \item Product launches
    \item Seasonal sales events
    \item Flash sales and limited-time promotions
    \item Emergency response requiring rapid inventory deployment
\end{itemize}




\section{Literature Review}\label{sec:literature-review}

The optimisation of supply chain distribution systems has been extensively studied in operations research, with mixed-integer programming (MIP) emerging as a dominant methodology for addressing complex logistics challenges.
In an effort to keep this literature review brief, and noting that the literature review is not the core of this report, we will only touch on the difficulties of inventory management surrounding Black Friday.

While the academic literature on Black Friday-specific optimisation is limited, industry reports highlight the unique challenges of promotional period supply chain management.
85\% of retailers are at least somewhat concerned about inventory shortages during BFCM (Black Friday and Cyber Monday)~\cite{anvyl2023black}, which emphasises the challenges retailers face around these periods.

In order to meet the targets being set by Black Friday's continued growth, supply chains are required to deliver high quantities of stock to specific locations with short deadlines~\cite{scmag2024black}.
This aligns with our modeling approach, which incorporates demand scaling factors ranging from 0.2x to 10.0x normal levels throughout the Black Friday week.

Recent industry trends show that `\textit{Early sales events like Black Friday and Cyber Monday have conditioned shoppers to start planning and purchasing well in advance}'~\cite{sc247consumer}, extending the planning horizon and requiring more sophisticated multi-period optimisation models.
This evolution in consumer behavior reinforces the need for flexible distribution strategies that can adapt to changing demand patterns over extended promotional periods.




\section{Data}\label{sec:data}

\subsection{Product Selection}\label{subsec:product-selection}
We selected three products representing typical Black Friday electronics categories:

\begin{table}[h]
\centering
\caption{Product specifications}
\begin{tabular}{lrrrr}
\toprule
Product & Retail Price & Shortage Cost & Holding Cost & Units/Truck \\
\midrule
Samsung Galaxy S25 Ultra & \$2,200~\cite{phonecost} & \$1,100 & \$44/day & 350~\cite{phonespecs} \\
Audio-Technica ATH-R50x & \$400~\cite{headphonescost} & \$200 & \$8/day & 60~\cite{headphonesspecs} \\
Samsung Galaxy Buds FE & \$200~\cite{earphonescost} & \$50 & \$10/day & 1,500~\cite{earphonesspecs} \\
\bottomrule
\end{tabular}\label{tab:table}
\end{table}

Product dimensions were sourced where possible from the manufacturer, otherwise from Amazon.
A small scaling up of the dimensions was applied to represent the space taken up by packaging.
Truck capacity allocations assume these products represent 1\% of total cargo, yielding the units per truck shown above.

\subsection{Cost Parameters}\label{subsec:cost-parameters}

\begin{itemize}
    \item \textbf{Holding costs} are based on 2\% daily rate for Black Friday inventory (2.5x normal rate):
    \begin{itemize}
        \item Standard calculation: based on reported industry standards~\cite{holdingcosts} that carry costs are typically 15\% - 30\% of the value of a company's inventory.
        Assuming this is an annual figure, this gives us around a 0.8\% of the item's retail price per day.
        \item Black Friday multiplier accounts for warehouse premiums and time sensitivity.
        \item Earphones have elevated holding cost (\$10 vs \$4 standard) due to obsolescence risk.
    \end{itemize}

    \item \textbf{Shortage costs} reflect lost profit margins:
    \begin{itemize}
        \item Standard calculation: based on reported industry standards~\cite{shortagecosts} that 50\% - 70\% are considered good margins, and that the shortage cost is close to the opportunity cost of a lost sale (ie. the margin)
        \item Phones and headphones: 50\% margin (industry standard for electronics)
        \item Earphones: 25\% margin due to heavy discounting
    \end{itemize}

    \item \textbf{Truck rental costs} from commercial providers:
    \begin{itemize}
        \item Weekly: \$1,100 for 7 days (50m³ truck)~\cite{truckcosts1}~\cite{truckcosts2}
        \item Daily: \$100-\$200 depending on demand
        \item Daily premium: 50\% surcharge over weekly rate
    \end{itemize}
\end{itemize}


\subsection{Demand Profiles}\label{subsec:demand-profiles}

We model identical demand patterns across all five retailers to isolate the effects of transportation costs and truck scheduling decisions.
The demand profiles reflect typical Black Friday consumer behavior:

\begin{table}[h]
\centering
\caption{Daily demand by product (units per retailer)}
\begin{tabular}{lrrrrrrr}
\toprule
Product & Mon & Tue & Wed & Thu & Fri (BF) & Sat & Sun \\
\midrule
Earphones (Good 0) & 100 & 50 & 30 & 10 & 1,000 & 600 & 400 \\
Headphones (Good 1) & 50 & 20 & 0 & 0 & 200 & 10 & 20 \\
Phones (Good 2) & 10 & 5 & 3 & 1 & 100 & 80 & 80 \\
\bottomrule
\end{tabular}\label{tab:table2}
\end{table}

Key demand characteristics:
\begin{itemize}
    \item \textbf{Earphones}: Steady decline from Monday to Thursday (100 down to 10 units), massive spike on Black Friday (1,000 units), gradual decline over weekend
    \item \textbf{Headphones}: No demand Wednesday-Thursday as consumers wait for deals, concentrated spike on Black Friday (200 units)
    \item \textbf{Phones}: Minimal pre-Black Friday demand, sustained weekend demand (80 units/day) as consumers continue to purchase through to Cyber Monday.
\end{itemize}

\subsection{Network Structure}\label{subsec:network-structure}
\begin{itemize}
    \item 5 warehouses representing distribution centers
    \item 5 retailers representing metropolitan locations
    \item Transportation costs varying in a range between \$0.40 to \$2.60 per unit, representing distances between stores and different characteristics of the goods (especially fragility)
    \item Maximum 10 trucks daily (driver availability constraint)
\end{itemize}

\subsection{Computational Scale Study}\label{subsec:computational-scale-study}

Beyond analyzing the specific Black Friday case, we investigate how our solution scales as we increase the number of warehouses, goods and retailers.
This is particularly important to consider, as our problem size grows exponentially across these three dimensions, and we want to ensure that we can model a realistic retailer scenario consisting of hundreds of goods and tens of warehouses and retailers.

Further information on the Computational Case Study, including scales studied and the methods for creating synthetic data is available in section~\ref{sec:computational-scale-study}.





\section{Model}\label{sec:model}

The following is the specification of the model for weekly truck rentals.
To see the modifications required to support daily truck rental, see Section~\ref{subsec:modifications-for-daily-truck-rental} \textit{Modifications for Daily Truck Rental}.
To simplify the model, we introduce two dummy decision variables \textit{carried\_retailer\_stock} and \textit{short\_retailer\_stock} to:
\begin{itemize}
    \item make calculation of holding and shortage costs easier,
    \item to enforce supply-demand balance in each period.
\end{itemize}

\subsection{Parameters}\label{subsec:parameters}
\begin{itemize}
    \item $W$: set of warehouses, $W = \{1, \hdots, n_w\}, n_w\in \mathbb{N}$.
    \item $G$: set of distinct goods, $G = \{1, \hdots, n_g\}, n_g \in \mathbb{N}$.
    \item $R$: set of retailers, $R = \{1, \hdots, n_r\}, n_r \in \mathbb{N}$.
    \item $P$: set of periods (number of days), $P = \{1, \hdots, n_p\}, n_p \in \mathbb{N}$.
    \item $K_T$: storage capacity of each truck, $K_T \in \mathbb{N}$.
    \item $R_T$: daily rental cost of each truck, $R_T \in \mathbb{N}$.
    \item $M_T$: maximum number of trucks that can be rented, $M_T \in \mathbb{N}$.
    \item $K_g$: size of good $g$, $K_g \in \mathbb{N}$.
    \item $H_g$: holding cost of good $g$, $H_g \geq 0$.
    \item $S_g$: shortage cost of good $g$, $S_g \geq 0$.
    \item $T_{w,r,g}$: transportation cost of good $g$ from warehouse $w$ to retailer $r$, $T_{w,r,g} \geq 0$.
    \item $D_{r,g,p}$: unit demand of good $g$ at retailer $r$ in period $p$, $D_{r,g,p} \geq 0$.
\end{itemize}

\subsection{Variables and Model}\label{subsec:variables-and-model}
\begin{itemize}
    \item $r$: total number of trucks rented for the week, $r \in \mathbb{N}$.
    \item $a_{w,p}$: number of trucks allocated to warehouse $w$ in period $p$, $a_{w,p} \in \mathbb{N}$.
    \item $t_{w,r,g,p}$: amount of good $g$ transported from warehouse $w$ to retailer $r$ in period $p$, $t_{w,r,g,p} \in \mathbb{N}$.
    \item $h_{r,g,p}$: dummy variable to track the amount of good $g$ held at retailer $r$ at start of period $p$ (ie. left from previous day), $h_{r,g,p} \in \mathbb{N}$.
    \item $s_{r,g,p}$: dummy variable to track the amount of good $g$ short of demand at retailer $r$ in period $p$, $s_{r,g,p} \in \mathbb{N}$.
\end{itemize}

Let

\begin{alignat}{2}
    C_t = & \sum_{w \in W}\sum_{r \in R}\sum_{g \in G}\sum_{p \in P} T_{w,r,g} t_{w,r,g,p}  & \forall w \in W, r \in R, g \in G, p \in P & \quad \text{(Total transport costs)} \\
    C_s = & \sum_{r \in R}\sum_{g \in G}\sum_{p \in P} S_{g} s_{r,g,p}  & \forall r \in R, g \in G, p \in P & \quad \text{(Total shortage costs)}\\
    C_h = & \sum_{r \in R}\sum_{g \in G}\sum_{p \in P} H_{g} h_{r,g,p}  & \forall r \in R, g \in G, p \in \{P \cup \{n_p + 1\}\} & \quad \text{(Total holding costs)}\\
    C_r = & \sum_{w \in W}\sum_{p \in P} R_T a_{w,p}  & \forall w \in W, p \in P & \quad \text{(Total truck rental costs)}
\end{alignat}

Then our model is

\begin{alignat}{2}
    \min \quad & C_t + C_s + C_h + C_r & &  \quad \text{(Minimise total cost of system)},\\
    \text{s.t.} \quad & \sum_{w\in W} a_{w,p} = r, & \quad & \forall p \in P, \quad \text{(Daily truck allocation = hired)},\\
    & \sum_{g \in G} K_g \sum_{r \in R} t_{w,r,g,p} \leq a_{w,p} * K_T, & \quad & \forall w \in W, p \in P, \quad \text{(Total sent less than capacity)},\\
    & h_{r,g,1} =0, & \quad & \forall r \in R, g \in G, \quad \text{(Retailers start with no stock)},\\
    & s_{r,g,p} \geq D_{r,g,p} - h_{r,g,p} - \sum_{w \in W} t_{w,r,g,p}, & \quad & \forall r \in R, g \in G, p \in P \quad \text{(Shortage defn)},\\
    & h_{r,g,(p+1)} = \sum_{w \in W} t_{w,r,g,p} - D_{r,g,p} + h_{r,g,p} + s_{r,g,p}, & \quad & \forall r \in R, g \in G, p \in P, \quad \text{(Held goods defn)},\\
    & r \geq 0 & \quad & \quad \text{(Scope of variables)},\\
    & a_{w,p} \geq 0 & \quad & \forall w \in W, p \in P, \quad \text{(Scope of variables)},\\
    & t_{w,r,g,p} \geq 0 & \quad & \forall w \in W, r \in R, g \in G, p \in P \quad \text{(Scope of variables)},\\
    & h_{r,g,p} \geq 0 & \quad & \forall r \in R, g \in G, p \in P \quad \text{(Scope of variables)},\\
    & s_{r,g,p} \geq 0 & \quad & \forall r \in R, g \in G, p \in P \quad \text{(Scope of variables)}.
\end{alignat}

\textbf{The objective function} minimises the total cost of the system.

\textbf{The first constraint} ensures all trucks allocated to warehouses within a period equals the total amount of trucks hired (ie. no unused trucks).

\textbf{The second constraint} ensures the total size of goods sent from each warehouse does not exceed the total capacity of all trucks assigned to that warehouse.

\textbf{The third constraint} specifies that each retailer starts with no stock on hand.

\textbf{The fourth constraint} ensures that the shortage of a particular good for a particular store in a particular period is at least equal to the unmet demand.
Since any additional units of shortage will contribute positively to the cost, combining this constraint with the objective function gives equality (ie. that shortage is equal to unmet demand).

\textbf{The fifth constraint} ensures the total units of a good left at a particular retailer in a particular period is equal to the amount held initially, plus the amount sent to the store, minus the demand of the day plus the shortage.
Equivalently, this constraint says that if shortage is zero, the change in held goods is equal to supply minus demand, and if shortage is positive then by constraint 4 the amount of held goods must be zero.


\subsection{Modifications for Daily Truck Rental}\label{subsec:modifications-for-daily-truck-rental}
For daily truck rental, we make two small modifications.
We must modify the variable $r$ to now allow for daily variations, and we must modify the first constraint to take into account this variability.
The variable $r$ becomes $r_p$:

\begin{itemize}
    \item $r_p$: total number of trucks rented for each period $p$, $r_p \geq 0$.
\end{itemize}

And our first constraint becomes

\begin{alignat}{2}
     & \sum_{w\in W} a_{w,p} = r_p, & \quad & \forall p \in P, \quad \text{(Daily truck allocation = hired)}
\end{alignat}

Note that we used separate instance data for the daily truck rental scenarios (\textit{daily\_trucks\_high\_cost}, \textit{daily\_trucks\_low\_cost}), and so the 50\% daily rental premium is already captured in a change to $R_T$.


\section{Solution strategy}\label{sec:solution-strategy}

As described in the above section, we propose an IP model to optimise the rental and scheduling plan.
Such a formulation is usually solved exactly using the branch-and-bound algorithm.
Therefore, in this project, we solve the proposed formulation using the Gurobi solver, which solves IP or MIP model using an improved branch-and-cut algorithm.
Compared with the standard branch-and-bound algorithm, this solver integrates some heuristics policy and valid cuts into the branch-and-bound algorithm, significantly accelerating the solving process.

However, this solver cannot handle our formulation when the size is large.
Specifically, we observe that the solver needs more than 100 seconds to solve the size of 5 warehouses, 30 types of goods, 15 retailers and 5 periods using our synthetic data.
To mitigate this issue, some techniques proposed in this section are applied to accelerate the solving process.
Based on appropriate experiments, we observe that these tricks are useful in some scales, helping the algorithm converge more quickly at certain problem sizes.

The challenges mainly include:
\begin{itemize}
    \item \textbf{Large Branch-and-bound Tree}: The proposed formulation is an integer programming problem, requiring more Gomory cuts to tighten the relaxation bound.
    As a result, the simplex method needs to be executed significantly more often than in binary programming, leading to a longer runtime.
    \item \textbf{Weak relaxation bound}: We note that the best bound provided by Gurobi is difficult to improve, which reduces the efficiency of the solver.
\end{itemize}

To mitigate these issues, we propose two strategies in this section - branch priority and cut search - which accelerate convergence.

\subsection{Branching strategy}\label{subsec:branching-strategy}
The branch priority is significant to the solving efficiency.
The branch-and-bound problem is far larger, especially when variables are integer, compared with binary case.
To mitigate this issue, we require the solver to branch significant variables with higher priority.

In our case, the priority of the number of trucks rented is assigned with the highest priority, followed by the scheduling variables, and then the shortage calculation.
The above process is implemented by setting the parameter \textbf{\textit{BranchPriority}} for each of the variables in Gurobi.

With this technique, the Gomory cuts derived by branching the truck number are relatively stronger than ones derived by branching other variables, tightening the relaxation bound.

\subsection{Active cuts search}\label{subsec:active-cuts-search}
We note that the best bound of the problem provided by the solver is hard to be improved, while the best incumbent is relatively easy to derive with the help of heuristics integrated into the solver.
Based on this observation, we attempt to add more valid cuts into the formulation to tighten the relaxation bound.
To implement this, we set the parameter \textbf{\textit{Cuts}} to \textbf{3} in Gurobi to encourage the solver to find more valid cuts.

Most of the cuts derived by this step are different from Gomory cuts, which are usually generated by exploring the structure of the problem.











\section{Results and analysis}\label{sec:results-and-analysis}

\subsection{Results}\label{subsec:results2}
Tables 1 to 6 presents the optimized total cost, daily cost and average truck utilization in each scenario.\\
\begin{table}[ht]
    \centering
    \caption{Minimized cost from the model}\label{tab:table3}
    \begin{tabularx}{1\textwidth}{
  | >{\centering\arraybackslash}X
  | >{\centering\arraybackslash}X
  | >{\centering\arraybackslash}X | >{\centering\arraybackslash}X | >{\centering\arraybackslash}X | >{\centering\arraybackslash}X | >{\centering\arraybackslash}X | }
  \hline
  Scenario & Number of Trucks (Each period) &Truck Hire Cost & Transport- -ation Cost & Shortage Cost & Holding Cost & Total Cost \\
  \hline
  High Rent, Daily & 5,2,2,10,10, 4,4 & 11100.0 & 11654.7 & 0.0 & 6464.0	& 29218.7 \\
  \hline
  Low Rent, Daily & 5,2,2,10,10, 4,5 & 5700.0 & 11621.2 & 0.0 & 6200.0 & 23521.2 \\
  \hline
  High Rent, Weekly & 10 & 14000.0	& 11518.8 & 0.0 & 6200.0 & 31718.8 \\
  \hline
  Low Rent, Weekly & 10 & 7000.0 & 11517.9	& 0.0 & 6200.0 & 24717.9 \\
  \hline
\end{tabularx}

\end{table}

\begin{table}[ht]
    \centering
    \caption{Minimized daily cost, when trucks are rented daily and expensive}\label{tab:table4}
    \begin{tabularx}{1\textwidth}{
  | >{\centering\arraybackslash}X
  | >{\centering\arraybackslash}X
  | >{\centering\arraybackslash}X | >{\centering\arraybackslash}X | >{\centering\arraybackslash}X | >{\centering\arraybackslash}X | }
  \hline
  Day & Truck Hire Cost & Transport- -ation Cost & Shortage Cost & Holding Cost & Total Cost \\
  \hline
  Day 1 & 1500.0 & 708.0 & 0.0 & 0.0 & 2208.0 \\
  \hline
  Day 2 & 600.0 & 355.4 & 0.0 & 0.0 & 955.4 \\
  \hline
  Day 3 & 3000.0 & 235.9 & 0.0 & 0.0 & 3235.9 \\
  \hline
  Day 4 & 3000.0 & 680.1 & 0.0 & 776.0 & 4456.1 \\
  \hline
  Day 5 & 1200.0 & 4754.4 & 0.0 & 5552.0 & 11506.4 \\
  \hline
  Day 6 & 1200.0 & 2827.0 & 0.0 & 64.0 & 4091.0 \\
  \hline
  Day 7 & 1200.0 & 2093.9 & 0.0 & 72.0 & 3365.9 \\
  \hline
\end{tabularx}

\end{table}

\begin{table}[ht]
    \centering
     \caption{Minimized daily cost, when trucks are rented daily and less expensive}\label{tab:table5}
    \begin{tabularx}{1\textwidth}{
  | >{\centering\arraybackslash}X
  | >{\centering\arraybackslash}X
  | >{\centering\arraybackslash}X | >{\centering\arraybackslash}X | >{\centering\arraybackslash}X | >{\centering\arraybackslash}X | }
  \hline
  Day & Truck Hire Cost & Transport- -ation Cost & Shortage Cost & Holding Cost & Total Cost \\
  \hline
  Day 1 & 750.0 & 708.0 & 0.0 & 0.0 & 1458.0 \\
  \hline
  Day 2 & 300.0 & 355.4 & 0.0 & 0.0 & 655.4 \\
  \hline
  Day 3 & 300.0 & 237.9 & 0.0 & 0.0 & 537.9 \\
  \hline
  Day 4 & 1500.0 & 733.7 & 0.0 & 712.0 & 2949.7 \\
  \hline
  Day 5 & 1500.0 & 4687.5 & 0.0 & 5488.0 & 11675.5 \\
  \hline
  Day 6 & 600.0 & 2830.9 & 0.0 & 0.0 & 3430.9 \\
  \hline
  Day 7 & 750.0 & 2067.8 & 0.0 & 0.0 & 2817.8 \\
  \hline
\end{tabularx}

\end{table}

\begin{table}[ht]
    \centering
    \caption{Minimized daily cost, when trucks are rented weekly and expensive}\label{tab:table6}
    \begin{tabularx}{1\textwidth}{
  | >{\centering\arraybackslash}X
  | >{\centering\arraybackslash}X
  | >{\centering\arraybackslash}X | >{\centering\arraybackslash}X | >{\centering\arraybackslash}X | >{\centering\arraybackslash}X | }
  \hline
  Day & Truck Hire Cost & Transport- -ation Cost & Shortage Cost & Holding Cost & Total Cost \\
  \hline
  Day 1 & 2000.0 & 704.0 & 0.0 & 0.0 & 2704.0 \\
  \hline
  Day 2 & 2000.0 & 325.5 & 0.0 & 0.0 & 2325.5 \\
  \hline
  Day 3 & 2000.0 & 228.0 & 0.0 & 0.0 & 2228.0 \\
  \hline
  Day 4 & 2000.0 & 616.2 & 0.0 & 712.0 & 3328.2 \\
  \hline
  Day 5 & 2000.0 & 4802.1 & 0.0 & 5488.0 & 12290.1 \\
  \hline
  Day 6 & 2000.0 & 2785.0 & 0.0 & 0.0 & 4785.0 \\
  \hline
  Day 7 & 2000.0 & 2058.0 & 0.0 & 0.0 & 4058.0 \\
  \hline
\end{tabularx}

\end{table}

\begin{table}[ht]
    \centering
    \caption{Minimized daily cost, when trucks are rented weekly and less expensive}\label{tab:table7}
    \begin{tabularx}{1\textwidth}{
  | >{\centering\arraybackslash}X
  | >{\centering\arraybackslash}X
  | >{\centering\arraybackslash}X | >{\centering\arraybackslash}X | >{\centering\arraybackslash}X | >{\centering\arraybackslash}X | }
  \hline
  Day & Truck Hire Cost & Transport- -ation Cost & Shortage Cost & Holding Cost & Total Cost \\
  \hline
  Day 1 & 1000.0 & 704.0 & 0.0 & 0.0 & 1704.0 \\
  \hline
  Day 2 & 1000.0 & 325.5 & 0.0 & 0.0 & 1325.5 \\
  \hline
  Day 3 & 1000.0 & 225.0 & 0.0 & 0.0 & 1225.0 \\
  \hline
  Day 4 & 1000.0 & 668.9 & 0.0 & 712.0 & 2380.9 \\
  \hline
  Day 5 & 1000.0 & 4751.5 & 0.0 & 5488.0 & 11239.5 \\
  \hline
  Day 6 & 1000.0 & 2785.0 & 0.0 & 0.0 & 3785.0 \\
  \hline
  Day 7 & 1000.0 & 2058.0 & 0.0 & 0.0 & 3058.0 \\
  \hline
\end{tabularx}

\end{table}
\\

\begin{table}[ht]
    \centering
    \caption{Average truck utilization comparison}\label{tab:table8}
    \begin{tabularx}{1\textwidth}{
  | >{\centering\arraybackslash}X
  | >{\centering\arraybackslash}X
  | }
  \hline
  Scenario & Average Truck Utilization \\
  \hline
  High Rent, Daily & 0.98 \\
  \hline
  Low Rent, Daily & 0.95 \\
  \hline
  High Rent, Weekly & 0.54 \\
  \hline
  Low Rent, Weekly & 0.54 \\
  \hline
\end{tabularx}

\end{table}

\subsection{Analysis}\label{subsec:analysis}

As the result shows, when the cost of renting trucks is high and trucks are scheduled on a daily basis, the model tries its best to meet all the demands (and succeeds) to avoid the massive shortage cost. Truck utilization for each warehouse at each period is close to 1 (See table 6), which is well expected for the condition where trucks are expensively rented daily, as wasting the capacity of trucks is very unprofitable. Holding cost is relatively low comparing to the truck renting cost and transportation cost, as trucks can be rented contingently and surplus in retailers can usually be avoided.

When the truck renting cost is low, it would be even more profitable to rent trucks contingently, and truck utilization can be more relaxed, as the ratio of profit gained by reducing the number of trucks rented to reducing carried goods is relatively lower compared to when trucks are expensive to rent (thus the model should favor more on reducing holding cost). This is reflected in the behaviour of the model as expected, where at period 6 (the seventh day), one more truck is rented comparing to the case when renting cost is high. This has enabled the warehouses to send less products in the period before (while still meeting all demands, thus in surplus), resulting a less total holding cost.

When trucks are rented on a weekly basis, the model still decides to minimize shortage cost as much as possible, even when trucks are expensive to hire. The main difference to the case where trucks are rented daily, is that during the periods when demands are low, there would be idle trucks. Therefore, truck utilization can be relaxed (see table 6), and the shipments can be made generally when demanded to reduce holding cost, and there is more freedom to decide on transportation schedule to reduce transportation cost. These effects are reflected on the results, that holding cost and transportation cost are lower than in the case where trucks are rented daily, and to a slight extent offsets the increased truck renting cost due to the change in renting schedule.

When the renting cost of trucks is low, it is interesting to see that the final cost only changes due to halved truck renting cost.
This indicates that the decision for transportation and supplying the retailers are already optimal to ensure that the shortage cost is minimal. In fact, holding costs only presents on day 4 and 5 (holding cost is induced from the surplus on previous day) in order to meet the peak demand on Friday.






\section{Computational scale study}\label{sec:computational-scale-study}
In this section, we compare the efficiency between our method and the benchmark (using Gurobi directly without any modifications).
The maximum runtime of Gurobi is limited to 100 seconds to meet the efficiency requirements of modern business operations.
The time spent on constructing the model isn't included since one can change the cost vector and constraint matrix of the existing model efficiently in practice, so we focus on the runtime of the model.
Computational experiments are performed on a computer equipped with an AMD Ryzen 7 6800H processor and 16 GB of RAM.
All algorithms are written in Python version 3.9.2 and Gurobi 10.0.1.

\subsection{Synthetic data}\label{subsec:synthetic-data}

Our collected dataset isn't enough to support the computational evaluation of a large-scale instance, we thus extend it with some modifications.
While some parameters are used directly, other parameters are generated using the random number API provided by Numpy.
We denote the uniform distribution with lower bound $b$ and upper bound $a$ as $U(a, b)$.
Note that a scaling factor is applied to simulate the Black Friday effect, similar to the last section.

\begin{table}[htbp]
    \centering
    \caption{Parameters setting}
    \label{tab:example}
    \begin{tabular}{ll}
        \toprule
        Parameter & Value \\
        \midrule
        good sizes & $U(1, 20)$ \\
        holding costs   & $U(20, 100)$  \\
        transportation costs   & $U(1, 4)$  \\
        penalty costs   & $U(1, 1.2) \times$ holding costs  \\
        basic demands         & $U(40, 100)$ \\
        \bottomrule
    \end{tabular}
\end{table}

In addition, we adjust the maximum number of trucks using the following expression to ensure that the optimisation problem is feasible but challenging:

$$
M_T = \left\lceil 1.2 \times \frac{\sum_{g \in G}\sum_{p \in P} \sum_{r \in R} D_{r,g,p}K_g}{K_T \times n_p} \right\rceil.
$$


\subsection{Results}\label{subsec:results}

In practice, the number of customer nodes and SKUs (goods) is typically larger than the number of warehouses and periods.
Therefore, we primarily focus on varying the number of customer nodes and SKUs to evaluate scalability, while making only slight adjustments to the number of warehouses and periods.

Each instance size is run 10 times to mitigate the randomness introduced by solver behavior.
The computational results are summarized in Table \ref{tab:method-comparison}.

\begin{table}[htbp]
    \centering
    \caption{Performance comparison}
    \label{tab:method-comparison}
    \begin{tabular}{lcc|cc}
        \toprule
        \multirow{2}{*}{\textbf{Scale}} & \multicolumn{2}{c|}{\textbf{Improved}} & \multicolumn{2}{c}{\textbf{Benchmark}} \\
        & Time (s) & Gap (\%) & Time (s) & Gap (\%) \\
        \midrule
        (5, 25, 15, 5)   & 0.38 & 0.00 & 0.38  & 0.00  \\
        (5, 25, 15, 10)   & \textbf{0.67} & 0.00 & 0.78  & 0.00  \\
        (5, 30, 15, 5)   & 0.25  & 0.00 & 0.19  & 0.00  \\
        (5, 35, 15, 5)  & \textbf{0.4}  & 0.00 & 0.55  & 0.00 \\
        (10, 25, 25, 5) & 0.53 & 0.00 & 0.43 & 0.00 \\
        (20, 60, 50, 5) & \textbf{2.8} & 0.00 & 3.99 & 0.00 \\
        (20, 100, 60, 5) & 6.93 & 0.00 & 5.81 & 0.00 \\
        (20, 120, 80, 5) & 10.03 & 0.00 & 9.56 & 0.00 \\
        \bottomrule
    \end{tabular}
    \captionsetup{justification=justified,singlelinecheck=false}
    \caption*{\footnotesize \textit{Note.} $(a, b, c, d)$ represents an instance with $a$ warehouses, $b$ goods, $c$ retailers, and $d$ periods.}
\end{table}

We observe that all instances can be solved to near optimality within 100 seconds.
However, our method performs better than the benchmark method in larger instances in solving efficiency, while similarly in small cases.
It is noteworthy that we set the parameter of \textbf{MIPGap} of the solver as 0.005, leading to the fact that the Gap value converges to 0.00 to prevent from tedious exploration of branch-and-bound tree.
Based on this observation, we suggest that one can set a small target \textbf{MIPGap} for the solver to derive a near optimal value (possibly optimal in most cases) within a relatively short duration.

Our strategies outperform the benchmark on certain medium-scale instances, such as $(20, 60, 50, 5)$, while the benchmark is faster  on others.
However, the differences aren't obvious.
In most cases, an optimal solution can be obtained within seconds, demonstrating the effectiveness of our model.
While both of them cannot dominate the other, in future work, we will attempt to explore more stable acceleration techniques.







\section{Conclusion and recommendations}\label{sec:conclusion-and-recommendations}

We propose an optimization model to help stakeholders determine the optimal plan of truck rental and scheduling plans to fulfill demands in different scenarios.
We focus on minimizing the total costs including rental, holding and penalty costs, while some practical constraints are considered.
With the support of an advanced solver, this IP model can be solved to optimality within 15 seconds under different instance sizes, fully meeting the efficiency requirements of modern business operations.

Furthermore, we attempt to propose some useful techniques to improve the solving efficiency.
Synthetic data are made to test the performance of the improved method and a benchmark provided by standard Gurobi.
Based on our numerical results, we notice that encourage the solver to branch significant variables with higher priority and search for valid cuts aggressively are beneficial to accelerate the convergence for medium scales.
However, its superiority isn't obvious under other scales.
In the future, we will explore other stable strategies to contribute to this field.

Some managerial insights are derived through numerical experiments supported by real datasets:
\begin{itemize}
    \item When truck rental is expensive and a daily pattern is applied, the model focuses on avoiding shortage costs by fully utilizing the rented trucks.
    \item When truck rental is cheap and a daily pattern is applied, the model seeks a more flexible transport plan to minimize holding costs once most demands are met.
    \item When truck rental is expensive and a weekly pattern is applied, some idle trucks are tolerated, while a more flexible delivery plan is preferred.
    \item When truck rental is cheap and a weekly pattern is applied, more trucks can be rented, allowing a more flexible delivery plan, which leads to a near-optimal rental and delivery strategy.
\end{itemize}

In summary, our model adapts its decisions to different cost structures, demonstrating its effectiveness in making strategic choices across varying environments.

In the future, this model can be improved by considering the following effects:
\begin{itemize}
    \item Several emerging innovations, such as parcel lockers and drones, have been developed to improve last-mile delivery.
    Exploring the integration of these technologies into the model could yield valuable contributions.
    \item The surge in demand during Black Friday is difficult to predict.
    In this report, a scaling factor was applied.
    However, developing a more robust model in the future could improve order fulfillment.
\end{itemize}







\section{Individual contributions}\label{sec:individual-contributions}
\begin{itemize}
    \item \textbf{Yitian Wang}
    \begin{itemize}
        \item Came up with the initial proposal that was submitted.
        \item Synthetic data creation.
        \item Solution strategy design and computational experiments design.
        \item Wrote Computational Case Study portion of the notebook.
        \item Wrote Solution Strategy (Section~\ref{sec:solution-strategy}) and Computation Case Study (Section~\ref{sec:computational-scale-study}) sections of the report.
    \end{itemize}

    \item \textbf{Dean Soste}
    \begin{itemize}
        \item Wrote Case Study notebook including data loading and model formulation.
        \item Conducted research into trucks and goods to hand-craft realistic demand patterns, truck storage capacities, holding and shortage costs.
        \item Co-ordinated group and facilitated meetings.
        \item Suggested the pivot towards a realistic business scenario (Black Friday) and increasing the complexity of the model toward a multi-period model.
        \item Wrote the Introduction (Section~\ref{sec:introduction}), Problem Definition (Section~\ref{sec:problem-definition}), Literature Review (Section~\ref{sec:literature-review}) and Data (Section~\ref{sec:data}) sections of the report.
    \end{itemize}

    \item \textbf{Tony Yue}
    \begin{itemize}
        \item Created the original dataset defined from the proposal.
        \item Created the presentation slides.
    \end{itemize}

    \item \textbf{Zhouyi Cheng}
    \begin{itemize}
        \item Wrote the Results and Analysis section of the report (Section~\ref{sec:results-and-analysis})
    \end{itemize}

    \item \textbf{Abdullah Alsuwat}
    \begin{itemize}
        \item Wrote the Conclusion and Recommendations section of the report (Section~\ref{sec:conclusion-and-recommendations})
    \end{itemize}

    \item \textbf{Sahar Bahalgardi}
    \begin{itemize}
        \item Wrote the Model section of the report (Section~\ref{sec:model})
    \end{itemize}
\end{itemize}






\begin{thebibliography}{99}

    \bibitem{anvyl2023black}
    Anvyl (2023).
    5 Black Friday \& Cyber Monday Supply Chain Challenges and How to Beat Them.
    Retrieved from \url{https://anvyl.com/blog/5-black-friday-cyber-monday-supply-chain-challenges/}

    \bibitem{scmag2024black}
    Supply Chain Magazine (2024).
    Black Friday: Reinforcing Supply Chains to Meet Demand.
    Retrieved from \url{https://supplychaindigital.com/digital-supply-chain/how-supply-chains-can-meet-black-friday-demand}

    \bibitem{sc247consumer}
    Supply Chain 24/7 (2024).
    How Consumer Trends Are Reshaping Black Friday and Supply Chains.
    Retrieved from \url{https://www.supplychain247.com/article/black-friday-new-consumer-retail-habits-supply-chain}

    \bibitem{phonecost}
    JB HiFi (2025).
    Samsung Galaxy S25 Ultra 512GB (Titanium Black).
    Retrieved from \url{https://www.jbhifi.com.au/products/samsung-galaxy-s25-ultra-512gb-titanium-black}

    \bibitem{phonespecs}
    Samsung (2025).
    Samsung Galaxy S25 Ultra 512GB (Titanium Black).
    Retrieved from \url{https://www.samsung.com/au/smartphones/galaxy-s25-ultra/specs/\#:~:text=Physical\%20specification,Gear\%20Support}

    \bibitem{headphonescost}
    JB Hifi (2025).
    Audio Technica ATH-R50X Professional Open Back Headphone.
    Retrieved from \url{https://www.jbhifi.com.au/products/audio-technica-ath-r50x-professional-open-back-headphone}

    \bibitem{headphonesspecs}
    Amazon (2025).
    Audio Technica ATH-R50X Professional Open Back Headphone.
    Retrieved from \url{https://www.amazon.com/Audio-Technica-Professional-Open-Back-Reference-Headphones/dp/B0DTTWF1Z9?th=1}

    \bibitem{earphonescost}
    JB HiFi (2025).
    Samsung Galaxy Buds FE (White).
    Retrieved from \url{https://www.jbhifi.com.au/products/samsung-galaxy-buds-fe-white}

    \bibitem{earphonesspecs}
    Amazon (2025).
    Samsung Galaxy Buds FE (White).
    Retrieved from \url{https://www.amazon.com.au/Samsung-Galaxy-Buds-FE-SM-R400NZWAEUA/dp/B0CPFH7FBM}

    \bibitem{holdingcosts}
    Zoho (2024).
    What is Carry Cost?
    Retrieved from \url{https://www.zoho.com/inventory/academy/inventory-management/what-is-carrying-cost.html}

    \bibitem{shortagecosts}
    UnleashedSoftware (2024).
    What is a Good Profit Margin?
    Retrieved from \url{https://www.unleashedsoftware.com/inventory-accounting-guide/what-is-a-good-profit-margin/}

    \bibitem{truckcosts1}
    Budget Trucks (2025).
    Victoria Weekly Rates on Trucks.
    Retrieved from \url{https://www.budgettrucks.com.au/en/offers/vic/weekly-rates}

    \bibitem{truckcosts2}
    Hertz Trucks (2025).
    Trucks with 50m3 capacity.
    Retrieved from \url{https://www.hertztrucks.com.au/bookings/choose-vehicle}


\end{thebibliography}



\end{document}
